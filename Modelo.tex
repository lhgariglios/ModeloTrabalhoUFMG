% Modelo Trabalho ABNT %

\documentclass[12pt,a4paper,oneside]{abntex2}
\usepackage[left=3cm,top=3cm,right=2cm,bottom=2cm]{geometry}
\usepackage[brazil]{babel} %lingua do texto
\usepackage[utf8]{inputenc} %permite caracteres como acentos e cedilhas
\usepackage{times} %usa a fonte times 
\setlength{\parindent}{1.25cm} %identação do parágrafo
\usepackage{indentfirst} % identa primeiro parágrafo % 
\usepackage{graphicx} %incluir figuras
\usepackage{lastpage} %para obter o número da última página
\pagenumbering{arabic} %Numeração de página

\author{Luiz Henrique Gariglio dos Santos}
\title{TÍTULO: subtítulo} 
\date{Ano}
\instituicao{UNIVERSIDADE FEDERAL DE MINAS GERAIS}
\local{Belo Horizonte}
\preambulo{ Trabalho de Conclusão de Curso apresentado ao Curso de Engenharia de Controle e Automação da Universidade Federal de Minas Gerais, como requisito parcial para o grau de bacharel em Engenharia de Controle e Automação.}
\orientador[Orientador: ]{Nome}
\coorientador[Coorientador: ]{Nome}
\tipotrabalho{Trabalho de Conclusão de Curso}

\renewcommand{\imprimircapa}{
	\begin{capa}
		\center 
		{\normalsize \textbf{\imprimirinstituicao} }\\[5cm]
		{\normalsize\imprimirautor}\\[4cm]
		{\normalsize\textbf{\imprimirtitulo}}\\
		\vfill
		{\normalsize\imprimirlocal}\\
		{\normalsize\imprimirdata}\\
		
	\end{capa}
}

\newcommand{\folhaderostonova}{
	\center 
	{\normalsize \imprimirautor } \\[5cm]
	{\normalsize \imprimirtitulo} \\[4.5cm]
	\hspace{.45 \textwidth} % posicionando a minipage
	\begin{minipage}{.5\textwidth}
		\imprimirpreambulo \\ \\
		\imprimirorientadorRotulo \imprimirorientador \\ \\
		\imprimircoorientadorRotulo \imprimircoorientador
	\end{minipage}
	\center
	\vfill
	{\normalsize \imprimirlocal } \\
	{\normalsize \imprimirdata}
}

%Corrigir títulos
\addto \captionsbrazil {\renewcommand{\listfigurename}{\normalsize\textbf{LISTA DE ILUSTRAÇÕES}}}
\addto \captionsbrazil {\renewcommand{\listtablename}{\normalsize\textbf{LISTA DE TABELAS}}}
\renewcommand{\listadesiglasname}{\normalsize\textbf{LISTA DE ABREVIATURAS E SIGLAS}}
\addto \captionsbrazil {\renewcommand{\contentsname}{\normalsize\textbf{SUMÁRIO}}}

\renewcommand{\ABNTEXchapterfontsize}{\normalfont\bfseries\normalsize} 
\renewcommand{\cftchapterfont}{\normalfont\bfseries} % Colocar negrito e fonte nos capítulos no sumário

\renewcommand{\ABNTEXsectionfontsize}{\normalsize\normalfont}
\renewcommand{\cftsectionfont}{\normalfont} % Tirar negrito das secoes no sumario

\renewcommand{\ABNTEXsubsectionfontsize}{\normalfont\bfseries\normalsize}
\renewcommand{\cftsubsectionfont}{\normalfont\bfseries} % Colocar negrito das subsecoes no sumario

\renewcommand{\ABNTEXsubsubsectionfontsize}{\normalfont\normalsize}
\renewcommand{\cftsubsubsectionfont}{\normalfont} % Tirar negrito subsubsection sumário

\begin{document}
	\imprimircapa
	\folhaderostonova
	\newpage
	\begin{fichacatalografica}
		\vspace*{\fill}
		%\hrule %linha horizontal
		\begin{table}[h]
			\footnotesize %tem tamanho 10 quando a letra do trabalho é 12 
			\begin{tabular}{|p{12.5cm}|}
				\hline
				\imprimirautor.\\
				
				\hspace{0.5cm} \imprimirtitulo / \imprimirautor, -- \imprimirdata.
				
				\hspace{0.5cm} \pageref{LastPage} p. \\
				
				\hspace{0.5cm} \imprimirorientadorRotulo \imprimirorientador.
				
				\hspace{0.5cm} \imprimirtipotrabalho -- Universidade Federal de Minas Gerais, Faculdade de Engenharia.\\
				
				\hspace{0.5cm}
				1. Palavra chave1.
				2. Palavra chave2.
				I. \imprimirtitulo.
				II. \imprimirorientador.
				III. Universidade Federal de Minas Gerais,Faculdade de Engenharia. \\
				\hline
			\end{tabular}
		\end{table}
		
	\end{fichacatalografica}
	
	\begin{errata}[\normalsize \textbf{ERRATA}]
		\flushleftright
		Santos, Luiz Henrique Gariglio. \textbf{\imprimirtitulo}. \imprimirdata. \imprimirtipotrabalho -- Faculdade de Engenharia, Universidade Federal de Minas Gerais, \imprimirlocal.
		
		\begin{table}[htb]
			\center
			\footnotesize
			\begin{tabular}{p{1.4cm}p{1cm}p{3cm}p{3cm}}
				\textbf{Página} & \textbf{Linha} & \textbf{Onde se lê}  & \textbf{Leia-se}
			\end{tabular}
		\end{table}
	\end{errata}
	
	%\includepdf{folhadeaprovacao_final.pdf}

	\begin{dedicatoria}% Opcional %
		\vspace*{\fill}
		\begin{flushright}
			Dedico esse modelo ao Eu do futuro...
		\end{flushright}
	\end{dedicatoria}

	\begin{agradecimentos}[\normalsize \textbf{AGRADECIMENTOS}]  % Opcional % corrigir
		À mim.
	\end{agradecimentos}

	\begin{epigrafe} % Opcional %
		\vspace*{\fill}
		\begin{flushright} % alinha o texto a direita %
			\textit{"Citação"\\Autor}
		\end{flushright} 
	\end{epigrafe}

	\begin{resumo}[\normalsize \textbf{RESUMO}]
		Resumo em português.
		
		\noindent \textbf{Palavras-chave}: 
	\end{resumo}
	
	\begin{resumo}[\normalsize \textbf{ABSTRACT}]
		
		\begin{otherlanguage*}{english}
			Resumo em inglês.
			
			\noindent \textbf{Keywords: }
		\end{otherlanguage*}

	\end{resumo}

	\listoffigures*
	\newpage
	\listoftables*
	\begin{siglas} % Lista de Siglas % % Opcional %
		\item[Sigla] Significado
	\end{siglas}
	\tableofcontents* \thispagestyle{empty} % Retira a numeração da página
	\mainmatter % Inicia a numeração das páginas
	
	\chapter{INTRODUÇÃO}
	\chapter{DESENVOLVIMENTO}
	\chapter{CONCLUSÃO}
	
	\begin{thebibliography}{1}
		\bibitem{ref1}  $\ll$ endereço do site $\gg$
	\end{thebibliography}

\end{document}